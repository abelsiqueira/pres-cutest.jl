\begin{frame}[fragile]
  \ctr{Exemplo de Slide}

  \begin{itemize}
    \item Para inserção de conteúdo, edite apenas o arquivo \texttt{body.tex}.
    \item Se quiser mudar a cor base, edite no arquivo \texttt{main.tex} a
    definição de \texttt{mybasecolor}.
    \item Existem três tipos básicos de slides: o \verb+\myframe[1]+, o
    \verb+\myframeblack[1]+, e o \verb+\myframecolor[2]+.
    \item Nenhum desses frames usa a opção \verb+fragile+. Se for necessário,
    crie um slide do modo tradicional.
  \end{itemize}
\end{frame}

\myframeblack{
  \ctrwhite{Exemplo de Slide}

  \begin{itemize}
    \item {\color{white}O slide anterior era um frame com fragile. O
      \texttt{myframe} é um slide normal, sem fragile.}
    \item {\color{white}Este é um \texttt{myframeblack}, bom para comandos do
      terminal e afins.}
    \item {\color{white}O \texttt{myframecolor}, aceita um argumento adicional
    da cor que deve ter o background. }
    \item {\color{white}Note que a cor de texto normal é black, então é preciso
    setar manualmente a cor branca.}
  \end{itemize}
}

\begin{frame}[fragile]
  \ctr{Exemplo de Slide}

  \begin{itemize}
    \item O título da página é feito com o comando \verb+\ctr+. O texto fica
      negrito e centralizado.
    \item Assim como o slide, temos o \verb+\ctrwhite+ e \verb+\ctrcolor+.
    \item Temos também o comando \verb+\cmd+; que gera um comando do terminal em
      amarelo; o comando \verb+\cmdinline+, que serve apenas como um \verb+\tt+
      de cor amarela; o comando \verb+\cmmt+, que gera um comentário inline no
      formato bash; o comando \verb+\bashgt+, que gera \bashgt; e o comando
      \verb+\ddash+, que gera \ddash.
  \end{itemize}
\end{frame}
